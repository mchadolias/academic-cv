\section{Research Experience}
  \vspace{1.25pt}
  \resumeSubHeadingListStart

    \resumeSubheading
      {Erlangen Centre for Astroparticle Physics (ECAP)}{Erlangen, Germany}
      {Master Thesis Student \href{\mthesislink}{(\textbf{Thesis})}}{Aug 2023 \textbf{--} Oct 2024, Full-time}
        \resumeItemListStart
            \resumeItem{Conducted the first feasibility study on tau neutrino appearance using ANTARES simulated data, evaluating the reconstruction quality of the events and providing the first estimate of the detector’s capabilities.}
            \resumeItem{Developed a scalable pipeline to convert over 500k ANTARES Monte Carlo event files to TTree-based ROOT files, reducing pre-processing time by 20\%, storage requirements by 35\%, and enabling parallelised cluster execution (NHR@FAU \& CC@Lyon).}
            \resumeItem{Analysed 200+ different oscillation scenarios with the SWIM software, accounting for the possible cut selection criteria, model parameters, and systematic uncertainties.}
            \resumeItem{Compared the tau appearance potential from ANTARES against KM3NeT/ORCA predicted sensitivities.}
        \resumeItemListEnd

    \resumeSubheading
      {Erlangen Centre for Astroparticle Physics (ECAP)}{Erlangen, Germany}
      {Graduate Research Assistant}{Nov 2022 \textbf{--} Apr 2023, Part-time}
        \resumeItemListStart
            \resumeItem{Co-developed and tested modules within a Snakemake-based MC production pipeline for ORCA events.}
            \resumeItem{Performed testing and diagnostics of pipeline jobs on NHR@FAU cluster, reducing execution errors and improving workflow reproducibility.}
            \resumeItem{Achieved 25\% CPU workload reduction by improving the SLURM profile, identifying inefficient parallelisation patterns, and optimising Snakemake rule resources.}
        \resumeItemListEnd

    \resumeSubheading
      {National Centre of Scientific Research Demokritos (NCSR)}{Athens, Greece}
      {Undergraduate Research Assistant \href{\internshiplink}{(\textbf{Report})}}{Nov 2021 \textbf{--} Jan 2022, Internship}
        \resumeItemListStart
            \resumeItem{Studied the reconstruction quality of simulated events focusing on the limited 6-string KM3NeT/ORCA detector configuration (ORCA6).}
            \resumeItem{Analysed $\nu_\mu$ interaction topologies across key metrics such as energy, zenith angle, inelasticity, and distance from the detector, among others.}
            \resumeItem{Assisted in the optical module integration and component testing.}
        \resumeItemListEnd    
    
    \resumeSubheading
      {SpaceDot Team - AcubeSAT Project}{Thessaloniki, Greece}
      {Undergraduate Researcher \& Sub-team Coordinator \href{\mthesislink}{(\textbf{Thesis})}}{Mar 2019 \textbf{--} Dec 2021, Part-time}
        \resumeItemListStart
            \resumeItem{Led the trajectory subsystem and co-authored a proposal for ESA's "Fly Your Satellite" programme}
            \resumeItem{Modelled the space environment for the mission with the OMERE, and estimated the radiation levels during the entire duration of the mission with the FASTRAD software, with this work being credited as my bachelor's thesis.}
            \resumeItemListEnd
  \resumeSubHeadingListEnd
