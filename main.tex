%-------------------------
% Resume in Latex
% Author : Michael Chadolias
% License : MIT
%------------------------
\documentclass[letterpaper,11pt]{article}

% Add your packages here
\usepackage{latexsym}
\usepackage[empty]{fullpage}
\usepackage{titlesec}
\usepackage{marvosym}
\usepackage[usenames,dvipsnames]{color}
\usepackage{verbatim}
\usepackage{enumitem}
\usepackage[hidelinks]{hyperref}
\usepackage{fancyhdr}
\usepackage{tabularx}
\usepackage{hyphenat}
\input{glyphtounicode}
\usepackage{xcolor}
\usepackage{multicol}

% ---------- Icons --------
\usepackage{fontawesome5}
\usepackage{academicons}
\definecolor{linkedincol}{HTML}{0077B5}
\definecolor{orcidlogocol}{HTML}{A6CE39}

% ---------- Bibliography Options --------
\usepackage[style=nature,backend=biber]{biblatex}
\DeclareLanguageMapping{english}{english-apa}
\addbibresource{publications.bib}  % Replace with your actual filename

% Add spacing between entries
\setlength\bibitemsep{1.1\baselineskip}

% Remove "ref_s" title
\defbibheading{bibliography}[]{}

% Language package
\usepackage[LGR, T1]{fontenc}
\usepackage[english]{babel}
\usepackage{textgreek}

%---------- Font Options ----------
% sans-serif
%\usepackage[sfdefault]{FiraSans}
%\usepackage[sfdefault]{roboto}
\usepackage[sfdefault]{noto-sans}
%\usepackage[default]{sourcesanspro}

% serif 
%\usepackage{CormorantGaramond}
%\usepackage{charter}


\pagestyle{fancy}
\fancyhf{} % clear all header and footer fields
\fancyfoot{}
\renewcommand{\headrulewidth}{0pt}
\renewcommand{\footrulewidth}{0pt}

% Adjust margins
\addtolength{\oddsidemargin}{-0.7in}
\addtolength{\evensidemargin}{-0.5in}
\addtolength{\textwidth}{1.2in}
\addtolength{\topmargin}{-.5in}
\addtolength{\textheight}{1.0in}

\urlstyle{same}

\raggedbottom
\raggedright
\setlength{\tabcolsep}{0in}

% Sections formatting
\titleformat{\section}{
  \vspace{-4pt}\scshape\raggedright\large
}{}{0em}{}[\color{black}\titlerule \vspace{-5pt}]

% Ensure that generate pdf is machine readable/ATS parsable
\pdfgentounicode=1

%-------------------------
% Custom commands
%\newcommand{\orcid}[1]{\href{https://orcid.org/#1}{\textcolor[HTML]{A6CE39}{\aiOrcid}}}

\newcommand{\resumeItem}[1]{
  \item\small{
    {#1 \vspace{-2pt}}
  }
}


\newcommand{\resumeSubheading}[4]{
  \vspace{-2pt}\item
    \begin{tabular*}{0.97\textwidth}[t]{l@{\extracolsep{\fill}}r}
      \textbf{#1} & #2 \\
      \textit{\small#3} & \textit{\small #4} \\
    \end{tabular*}\vspace{-7pt}
}


\newcommand{\resumeSubSubheading}[2]{
    \vspace{-2pt}\item
    \begin{tabular*}{0.97\textwidth}{l@{\extracolsep{\fill}}r}
      \textit{\small#1} & \textit{\small #2} \\
    \end{tabular*}\vspace{-7pt}
}


\newcommand{\resumeEducationHeading}[6]{
  \vspace{-2pt}\item
    \begin{tabular*}{0.97\textwidth}[t]{l@{\extracolsep{\fill}}r}
      \textbf{#1} & #2 \\
      \textit{\small#3} & \textit{\small #4} \\
      \textit{\small#5} & \textit{\small #6} \\
    \end{tabular*}\vspace{-5pt}
}


\newcommand{\resumeProjectHeading}[2]{
    \vspace{-2pt}\item
    \begin{tabular*}{0.97\textwidth}{l@{\extracolsep{\fill}}r}
      \small#1 & #2 \\
    \end{tabular*}\vspace{-7pt}
}


\newcommand{\resumeOrganizationHeading}[4]{
  \vspace{-2pt}\item
    \begin{tabular*}{0.97\textwidth}[t]{l@{\extracolsep{\fill}}r}
      \textbf{#1} & \textit{\small #2} \\
      \textit{\small#3}
    \end{tabular*}\vspace{-7pt}
}

\newcommand{\resumeSubItem}[1]{\resumeItem{#1}\vspace{-4pt}}

\renewcommand\labelitemii{$\vcenter{\hbox{\tiny$\bullet$}}$}

\newcommand{\resumeSubHeadingListStart}{\begin{itemize}[leftmargin=0.15in, label={}]}
\newcommand{\resumeSubHeadingListEnd}{\end{itemize}}
\newcommand{\resumeItemListStart}{\begin{itemize}}
\newcommand{\resumeItemListEnd}{\end{itemize}\vspace{-5pt}}

\begin{document}

%---------- HEADING ----------

\begin{center}
    \textbf{\Huge \scshape Michael Chadolias} \\ \vspace{3pt}
    \small
    \faMobile \hspace{.5pt} \href{tel:00306942201140}{+30 69 4220 1140}
    $|$
    \faAt \hspace{.5pt} \href{mailto:mchadolias@km3net.de}{mchadolias@km3net.de}
    $|$
    \faLinkedinSquare \hspace{.5pt} \href{https://www.linkedin.com/in/michael-chadolias/}{LinkedIn}
    $|$
    \faGithub \hspace{.5pt} \href{https://github.com/mchadolias}{GitHub}
    $|$
    \faMapMarker \hspace{.5pt} \href{https://www.google.com/maps/place/Erlangen/@49.5892592,10.9020837,12z/data=!3m1!4b1!4m6!3m5!1s0x47a1f8c7d57c34a1:0x41eda32beb5c7d0!8m2!3d49.5896744!4d11.0119611!16zL20vMDFjel8x?entry=ttu}{Erlangen, Germany}
\end{center}

% ---- Add extra space between header and sections (as per Elena's comment)
Physics graduate specialising in astroparticle physics with hands-on experience in neutrino detection, large-scale workflow pipelines, and machine learning for high-energy physics. Included in the author list of 8 peer-reviewed publications as a member of the KM3NeT collaboration. Looking to pursue a PhD in neutrino physics.
\vspace{1.5pt}

% ---------- RESEARCH INTERESTS -------

\section{Research Interests}
  \vspace{2pt}
  \resumeSubHeadingListStart
    \small{\item{
        {Neutrino physics, Neutrino Oscillations, Cosmic ray physics, Machine Learning, Data Science and Statistical Methods.} 
    }}
  \resumeSubHeadingListEnd

%----------- EDUCATION -----------
\section{Education}
  \vspace{3pt}
  \resumeSubHeadingListStart
    
    \resumeEducationHeading
      {Friedrich-Alexander-Universität Erlangen-Nürnberg
      }{Erlangen, Germany}
      {M.Sc. in Physics (Specialisation: Astrophysics and astroparticle physics)}{Apr 2022 \textbf{--} February 2025}
      {\textbf{Grade: 1.6/5.00}}{}
        \resumeSubHeadingListStart
        \small{\item{
             \textbf{Master Thesis:} Detection of low-energy tau neutrinos with the
                    ANTARES neutrino telescope - a feasibility study}}
        \small{\item{
             \textbf{Research:} Active participation in the ANTARES/KM3NeT collaboration through ECAP}}
        \small{\item{
             \textbf{Research Output:} In the author list of 8 peer-reviewed papers}}
        \resumeSubHeadingListEnd

    \resumeSubheading
    {Aristotle University of Thessaloniki (AUTh)
      }{Thessaloniki, Greece}
      {B.Sc. Physics;
      \textbf{Grade: 7.84/10.00}}{Oct 2018 \textbf{--} Mar 2022}
        \resumeSubHeadingListStart
        \small{\item{
             \textbf{Bachelor Thesis:} Monte Carlo Radiation Transport Problem - Radiation Analysis for the AcubeSAT}}
        \small{\item{
             \textbf{Internship Title:} ORCA - Energy Reconstruction Studies}}
        \small{\item{
             \textbf{Teaching Assistant:} HM 131: Systems Reliability under Prof. Alkiviadis Hatzopoulos}}
        \small{\item{
            Among the top 10\% of students during my graduation period}}  
        \resumeSubHeadingListEnd

      
  \resumeSubHeadingListEnd

%----------- EXPERIENCE -----------

\section{Research Experience}
  \vspace{3pt}
  \resumeSubHeadingListStart

    \resumeSubheading
      {Erlangen Centre for Astroparticle Physics (ECAP)}{Erlangen, Germany}
      {Master Thesis Student}{Aug 2023 \textbf{--} Oct 2024, Full-time}
        \resumeItemListStart
            \resumeItem{Conducted the first feasibility study on tau neutrino appearance using the full duration of ANTARES, evaluating the reconstruction quality of the events and providing the first estimate of the detector’s capabilities.}
            \resumeItem{Developed a scalable pipeline to convert over 500k ANTARES Monte Carlo event files to TTree-based ROOT files, reducing pre-processing time by 20\%, storage requirements by 35\%, and enabling parallelised cluster execution (NHR@FAU \& CC@Lyon).}
            \resumeItem{Analysed 200+ different oscillation scenarios with the SWIM software, accounting for the possible cut selection criteria, model parameters, and systematic uncertainties.}
            \resumeItem{Compared the tau appearance potential from ANTARES against KM3NeT/ORCA predicted sensitivities.}
        \resumeItemListEnd

    \resumeSubheading
      {Erlangen Centre for Astroparticle Physics (ECAP)}{Erlangen, Germany}
      {Research Assistant}{Nov 2022 \textbf{--} Apr 2023, Part-time}
        \resumeItemListStart
            \resumeItem{Co-developed and tested modules within a Snakemake-based MC production pipeline for ORCA events.}
            \resumeItem{Performed testing and diagnostics of pipeline jobs on NHR@FAU cluster, reducing execution errors and improving workflow reproducibility.}
            \resumeItem{Achieved 25\% CPU workload reduction by improving the SLURM profile, identifying inefficient parallelisation patterns, and optimising Snakemake rule resources.}
        \resumeItemListEnd

    \resumeSubheading
      {National Centre of Scientific Research Democritus}{Athens, Greece}
      {Undergraduate Research Assistant}{Nov 2021 \textbf{--} Jan 2022, Internship}
        \resumeItemListStart
            \resumeItem{Studied the reconstruction quality of simulated events focusing on the limited 6 string KM3NeT/ORCA detector configuration (ORCA6).}
            \resumeItem{Analysed $\nu_\mu$ interaction topologies across key metrics such as energy, zenith angle, inelasticity, and distance from the detector, among others.}
            \resumeItem{Assisted in the optical module integration and component testing.}
        \resumeItemListEnd    
    
    \resumeSubheading
      {SpaceDot - AcubeSAT Project}{Thessaloniki, Greece}
      {Space Environment Analyst - Coordinator of the Trajectory Subsystem}{Mar 2019 \textbf{--} Dec 2021, Part-time}
        \resumeItemListStart
            \resumeItem{Led the trajectory subsystem and co-authored a proposal for ESA's "Fly your satellite" programme}
            \resumeItem{Modelled the space environment for the mission with the OMERE, and estimated the radiation levels during the entire duration of the mission with the FASTRAD software, with this work being credited as my bachelor's thesis.}
            \resumeItemListEnd
  \resumeSubHeadingListEnd


%----------- RELEVANT COURSEWORK -----------

\section{Relevant Coursework}
  \vspace{2pt}
  \resumeSubHeadingListStart
    \small{\item{
        \textbf{Specialization Lectures:}{ Particle Physics, Astroparticle Physics, Neutrino Physics \& Neutrino Astronomy} \\ \vspace{3pt}
        
        \textbf{Minor coursework:}{ Computational Physics and Numerical Methods, Methods of Data Analysis}
    }}
  \resumeSubHeadingListEnd

% % ---------- Teaching Experience -------
% \section{Teaching Experience}
% \resumeSubHeadingListStart
%     \resumeSubheading
%     {HM 131: Systems Reliability under Prof. Alkiviadis Hatzopoulos}{AUTh}
%     {Teaching Assistant}{Oct 2020 - Feb 2021}
%     \resumeItem{Designed and evaluated an experimental exercise, regarding the reliability of electrical components in the space, using the OMERE software.}
% \resumeSubHeadingListEnd


%----------- SKILLS -----------

\section{Skills}
  \vspace{2pt}
  \resumeSubHeadingListStart
    \small{\item{
        
        \textbf{Languages:}{ Python, C/C++, Bash, SQL} \\ \vspace{3pt}

        \textbf{Frameworks \& Tools:}{ ROOT, SWIM, TensorFlow, PyTorch, GraphNet, JAX} 
        
        \textbf{Computing Methods:}{ High-Performance Computing (Slurm, UNIX), Workflow Management (Snakemake), Version Control (Git), CI/CD (Github Actions), Containerization (Docker, Apptainer)} \\ \vspace{3pt}
        
        % \textbf{Methodologies:}{ Object Oriented Programming, Functional Programming, High Performance Programming} \\ \vspace{3pt}

        % \textbf{Mathematics:}{Linear Algebra, Numerical Methods, } \\ \vspace{3pt}

        \textbf{Foreign Languages:} {Greek (Mother), English(C2), German(B2), Italian(A1)} \\ \vspace{3pt}

        \textbf{Soft Skills:}{ Team player, analytical thinker, problem-solving, adaptability} \\ \vspace{3pt}
    }}
  \resumeSubHeadingListEnd



%----------- PROJECTS -----------

% \section{Projects}
%     \vspace{3pt}
%     \resumeSubHeadingListStart
      
%       \resumeProjectHeading
%         {\textbf{Filters and Fractals} $|$ \emph{\href{https://github.com/arasgungore/filters-and-fractals}{\color{blue}GitHub}}}{}
%           \resumeItemListStart
%             \resumeItem{A C project which implements a variety of image processing operations that manipulate the size, filter, brightness, contrast, saturation, and other properties of PPM images from scratch and recursive fractal generation functions to model popular fractals including Mandelbrot set, Julia set, Koch curve, Barnsley fern, and Sierpinski triangle.}
%           \resumeItemListEnd
      
%       \resumeProjectHeading
%         {\textbf{Chess Bot} $|$ \emph{\href{https://github.com/arasgungore/chess-bot}{\color{blue}GitHub}}}{}
%           \resumeItemListStart
%             \resumeItem{A C++ project in which you can play chess against an AI with a specified decision tree depth that uses alpha-beta pruning algorithm to predict the optimal move. Aside from basic moves, this mini chess engine also implements chess rules such as castling, en passant, fifty-move rule, threefold repetition, and pawn promotion.}
%           \resumeItemListEnd
      
%       \resumeProjectHeading
%         {\textbf{DS\&A Projects} $|$ \emph{\href{https://github.com/arasgungore/CMPE250-projects}{\color{blue}GitHub}}}{}
%           \resumeItemListStart
%             \resumeItem{Five Java projects that apply DS\&A concepts such as discrete-event simulation using priority queues, Dijkstra's shortest path algorithm, Prim's algorithm to find the minimum spanning tree, Dinic's algorithm for maximum flow problems, and weighted job scheduling with dynamic programming to real-world problems.}
%           \resumeItemListEnd
      
%     \resumeSubHeadingListEnd


%----------- CERTIFICATES -----------

\section{Conferences \& Workshops}
\resumeSubHeadingListStart
      \resumeSubheading
      {Analysis Reproducibility}
      {Mar 2025}{Participant}{Online}
      \resumeSubheading
      {12th HEP C++ Course and Hands-on Training - The Essentials}
      {Mar 2025}{Participant}{Online}
      
      \resumeSubheading
      {KM3NeT Bootcamp}
      {Dec 2024}{Participant}{Erlangen, DE}
      
      \resumeSubheading
      {DPG Spring Meeting}
      {Mar 2024}{Speaker: Feasibility study of tau appearance measurement with the ANTARES neutrino telescope (T 29.2)}{Karlruhe, DE}

      \resumeSubheading
      {KM3NeT Oscillation WG Meeting}
      {Dec 2023}{Speaker: Feasibility study of tau appearance measurement with the ANTARES neutrino telescope}{Erlangen, DE}

      \resumeSubheading
      {KM3NeT Bootcamp}
      {Oct 2023}{Participant}{Online}
      
      \resumeSubheading
      {GraphNeT III - Workshop}
      {May 2023}{Participant}{Sandvig, DN}
\resumeSubHeadingListEnd



%----------- ORGANIZATIONS -----------

% \section{Organizations}
  % \resumeSubHeadingListStart
    
    % \resumeOrganizationHeading
      % {Institute of Electrical and Electronics Engineers (IEEE)}{Feb 2022 -- Present}{Student Member}
    
  % \resumeSubHeadingListEnd

% ---------- Volunteering -----
\section{Volunteering}
    \resumeSubHeadingListStart
    \resumeSubheading
      {Long Night of Sciences, FAU}{Erlangen, Germany}
      {Volunteer - Neutrino Group ECAP}{October 2023}
        \resumeItemListStart
            \resumeItem{Presented neutrino detection concepts to 1100+ visitors at ECAP through posters and demos, as well as showcasing the group’s contributions to neutrino physics \& astronomy.}
        \resumeItemListEnd
        \resumeSubheading
      {AUTh - CERN Masterclass}{Thessaloniki, Greece}
      {Volunteer}{Mar 2021}
        \resumeItemListStart
            \resumeItem{Guided 10 high school students through ATLAS event identification exercises during the AUTh–CERN Masterclass.}
        \resumeItemListEnd
    \resumeSubheading
      {Physicists of Aristotle university of Thessaloniki (PATH) }{Thessaloniki, Greece}
      {Group coordinator}{Nov 2017 \textbf{--} Jun 2019}
        \resumeItemListStart
            \resumeItem{Organized the speaker list and conference logistics to highlight undergraduate research in physics for the second iteration of the $\Sigma\Pi E\Phi$ conference.}
            \resumeItem{Facilitated student visits to research labs through the 'Labs \& Paths' event, helping fellow students engage with the research conducted in the department.}
    \resumeItemListEnd    
    % \resumeSubheading
    %   {TeDX Thessaloniki}{Thessaloniki, Greece}
    %   {Volunteer}{Mar 2017 \textbf{--} May 2019}
    %     \resumeItemListStart
    %         \resumeItem{ Primarily assisted as a sound technician assistant, but also helped backstage.}
    % \resumeItemListEnd
    \resumeSubHeadingListEnd


%----------- HOBBIES -----------

\section{Hobbies}
  \resumeSubHeadingListStart
    \small{\item{Chess (Ex-professional player, national arbiter), Basketball, Hiking, D\&D fan, Book-club member}}
  \resumeSubHeadingListEnd



% %----------- REFERENCES -----------

% \section{References}
%   \vspace{2pt}
%   \resumeSubHeadingListStart
%     \item \textit{References available upon request.}
%     %\item{PD. Dr. Thomas Eberl (ECAP) \tab Dr. Aikaterini Tzamarioudaki (NCSR Democritus)}
%   \resumeSubHeadingListEnd

%----------- Publications -----------
\section{Publications}
\vspace{0.2cm}
\nocite{*}
\printbibliography

%-------------------------------------------
\end{document}